\chapter{Tecnologias}\label{cap_tecnologias}

\section{Tecnologias}
Neste trabalho, conforme citado inicialmente, foram usadas as tecnologias de Java, Swing, Jena, GIT e a IDE IntelliJ.

\subsection{Java e Swing}
Dentre estas tecnologias, a linguagem Java foi usada para as telas (especificamente o Swing) e regras de negócio da aplicação. O desenvolvimento em Java só é possível devido às classes para manipulação e execução dos resultados advindos da query, além da própria biblioteca Jena.

\subsection{IDE Intellij}
Também foi utilizada IDE IntelliJ para facilitar o desenvolvimento, "O IDE é um programa de computador, geralmente utilizado para aumentar a produtividade dos desenvolvedores de software, bem como a qualidade desses produtos. Podem auxiliar, através de ferramentas e características, na redução de erros e na aplicação de técnicas como o RAD (Rapid Application Development)".

\subsection{GitHub}
Também foi feito o uso da ferramenta de controle de versão GitHub para gerenciar os arquivos de desenvolvimento e também deste relatório.